%------------------------------------------------------------------------------%
\documentclass[fleqn,utf8,aspectratio=169,12pt,ignorenonframetext]{beamer}

\usepackage{calculo_varias_variaveis-1}

\title{Apresentação da Disciplina}

%------------------------------------------------------------------------------%
\begin{document}

\addtitle

%------------------------------------------------------------------------------%
\section{Introdução}

%------------------------------------------------------------------------------%
\begin{frame}{Disciplina}
   Disciplina: \emph{\large Cálculo de Funções de Várias Variáveis - I}\\[7mm]
   Professor: \emph{\large Luis A. D'Afonseca}
\end{frame}

%------------------------------------------------------------------------------%
\begin{frame}{Disciplina}
\begin{itemize}[<+->]
  \item Aula semi invertida
  \item Estude o conteúdo com antecedência
  \item As apresentações são um resumo da teoria -- \emph{Estude o livro}
  \item Traga papel e lápis para fazer os exercícios em aula
  \item Estude o conteúdo seriamente entre uma aula e outra
  \item Não deixe o conteúdo acumular
  \item Não maratone o estudo de Matemática
  \item Preencha as lacunas do seu conhecimento de Matemática \\
        \href{https://pt.khanacademy.org}{Khan Academy}
\end{itemize}
\end{frame}

%------------------------------------------------------------------------------%
\section{Conteúdo}


%------------------------------------------------------------------------------%
\begin{frame}{Disciplina}
  Cálculo de Funções de Várias Variáveis I
  \begin{enumerate}[<+->]
    \item Curvas parametrizadas, coordenadas polares e superfícies quádricas
    \item Funções de várias variáveis
    \item Introdução aos números complexos
  \end{enumerate}
\end{frame}

%------------------------------------------------------------------------------%
\begin{frame}{Curvas}
  Curvas parametrizadas, coordenadas polares e superfícies quádricas
  \begin{enumerate}[<+->]
    \item Curvas parametrizadas no plano e no espaço
    \begin{enumerate}
      \item Definição
      \item Principais exemplos
      \item Vetor tangente
    \end{enumerate}
    \item Coordenadas polares
    \item Equações e esboço das principais superfícies quádricas via cortes
  \end{enumerate}
\end{frame}

%------------------------------------------------------------------------------%
\begin{frame}{Funções}
  Funções de várias variáveis
  \begin{enumerate}[<+->]
    \item Conceito, gráfico, curvas de nível, superfícies de nível
    \item Limites e continuidade
    \item Derivada parcial, derivadas de maior ordem
    \item Plano tangente, aproximação linear, diferenciabilidade
    \item Regra da cadeia, derivada implícita
    \item Derivada direcional, vetor gradiente
    \item Máximos e mínimos, pontos críticos, problemas de otimização
    \item Máximos e mínimos com restrições, multiplicadores de Lagrange
  \end{enumerate}
\end{frame}

%------------------------------------------------------------------------------%
\begin{frame}{Complexos}
  Introdução aos números complexos
  \begin{enumerate}[<+->]
    \item Introdução aos números complexos
    \item Interpretação vetorial
    \item Operações: adição, subtração, produto e razão
    \item Forma polar
    \item Potência
    \item Raízes $n$-ésimas de números complexos
    \item Fórmula de Euler
  \end{enumerate}
\end{frame}

%------------------------------------------------------------------------------%
\section{Lista Mínima}

%------------------------------------------------------------------------------%
\begin{frame}{Lista Mínima}
  \begin{enumerate}
    \item Ler Capítulo 1 do Thomas vol. 1 -- Funções
    \item Ler Capítulo 2 do Thomas vol. 1 -- Limites e Continuidade
    \item Estudar \href{https://pt.khanacademy.org}{Khan Academy}
    \begin{enumerate}
      \item \href{https://pt.khanacademy.org/math/precalculus}{Pré-cálculo}
      \item \href{https://pt.khanacademy.org/math/differential-calculus}{Cálculo diferencial}
      \item \href{https://pt.khanacademy.org/math/multivariable-calculus}{Cálculo multivariável}
    \end{enumerate}
  \end{enumerate}

  \vfill
  Atenção: A prova é baseada no livro, não nas apresentações
\end{frame}

%------------------------------------------------------------------------------%
\end{document}
%------------------------------------------------------------------------------%
