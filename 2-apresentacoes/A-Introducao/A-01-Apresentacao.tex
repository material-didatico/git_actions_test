%------------------------------------------------------------------------------%
\documentclass[fleqn,utf8,aspectratio=169,12pt]{beamer}

\usepackage{calculo_varias_variaveis-1}

\title{Apresentação da Disciplina}

%------------------------------------------------------------------------------%
\begin{document}

\addtitle

%------------------------------------------------------------------------------%
\section{Introdução}

%------------------------------------------------------------------------------%
\begin{frame}{Disciplina}
   Disciplina: \emph{\large Cálculo de Funções de Várias Variáveis - I}\\[7mm]
   Professor: \emph{\large Luis A. D'Afonseca}
\end{frame}

%------------------------------------------------------------------------------%
\begin{frame}{Disciplina}
\begin{itemize}[<+->]
  \item Aula semi invertida
  \item Estude o conteúdo com antecedência
  \item As apresentações são um resumo da teoria -- \emph{Estude o livro}
  \item Traga papel e lápis para fazer os exercícios em aula
  \item Estude o conteúdo seriamente entre uma aula e outra
  \item Não deixe o conteúdo acumular
  \item Não maratone o estudo de Matemática
  \item Preencha as lacunas do seu conhecimento de Matemática \\
        \href{https://pt.khanacademy.org}{Khan Academy}
\end{itemize}
\end{frame}

\framedgraphic{Teste}{transformacao}{}

%------------------------------------------------------------------------------%
\end{document}
%------------------------------------------------------------------------------%
