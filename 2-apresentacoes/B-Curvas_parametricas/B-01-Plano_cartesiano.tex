%------------------------------------------------------------------------------%
\documentclass[fleqn,utf8,aspectratio=169,12pt,ignorenonframetext]{beamer}

\usepackage{calculo_varias_variaveis-1}

\usepackage{tikz}
\usetikzlibrary{angles}

\title{Plano Cartesiano}

%------------------------------------------------------------------------------%
\begin{document}

\addtitle

%------------------------------------------------------------------------------%
\section{Sistema de Coordenadas Cartesianas}

%------------------------------------------------------------------------------%
\begin{frame}{Sistema de Coordenadas Cartesianas}
  \begin{enumerate}[<+->]
    \item Cada ponto do plano é associado a um par de números reais $(x, y)$
    \item Cada ``figura'' no plano corresponde a um subconjunto de $\R^2$
    \item Na orientação convencional, $x$ aponta para a direita e $y$ para cima
  \end{enumerate}
\end{frame}

%------------------------------------------------------------------------------%
\begin{frame}{Sistema de Coordenadas Cartesianas}
  \centering
  \input{plano_cartesiano}
\end{frame}

%------------------------------------------------------------------------------%
\begin{frame}{Gráfico de uma Função Real}
  Conjunto de pontos $(x, y) \in \R^2$ tais que \(y = f(x)\)

  \pause
  \vspace{\baselineskip}
  Para cada $x$, no domínio, existe \emph{um único} $y$
\end{frame}

%------------------------------------------------------------------------------%
\begin{frame}{Gráfico de uma Função Real}
  \centering
  \input{grafico_funcao_real}
\end{frame}

%------------------------------------------------------------------------------%
\begin{frame}{Soluções de uma Equação}

  Qualquer expressão da forma
  \[
    G(x, y) = 0
  \]

  \pause
  \vspace{\baselineskip}
  Queremos \emph{todos os pares} $(x, y)$ que satisfazem a equação

  \pause
  \vspace{\baselineskip}
  Se faltar um, a solução está incompleta e portanto errada
\end{frame}

%------------------------------------------------------------------------------%
\begin{frame}{Soluções de uma Equação}
  \centering
  \input{grafico_equacao}
\end{frame}

%------------------------------------------------------------------------------%
\begin{frame}{Curvas Paramétricas}
  Funções de $\R$ em $\R^n$
  \[
    \left(\begin{array}{>{\,}c<{\,}}
      x \\[1mm]
      y
    \end{array}\right)
    =
    \left(\begin{array}{>{\,}c<{\,}}
      f(t) \\[1mm]
      g(t)
    \end{array}\right)
  \]

  \pause
  \vspace{\baselineskip}
  Podemos pensar na variável como o tempo e a curva como uma
  trajetória

\end{frame}

%------------------------------------------------------------------------------%
\begin{frame}{Curvas Paramétricas}
  \centering
  \input{grafico_parametricas}
\end{frame}

%------------------------------------------------------------------------------%
\begin{frame}{Desigualdades}
  Desigualdades geralmente representam regiões no plano (ou espaço)
\end{frame}

%------------------------------------------------------------------------------%
\begin{frame}{Desigualdades}
  \centering
  \input{grafico_desigualdade}
\end{frame}

%------------------------------------------------------------------------------%
%\section{Lista Mínima}

%------------------------------------------------------------------------------%
%\begin{frame}{Lista Mínima}
%  Cálculo Vol. 2 do Thomas 12\textsuperscript{a} ed. -- Seção 11.3
%  \begin{enumerate}
%    \item Estudar o texto da seção
%    \item Resolver os exercícios: 1, 2, 6, 7, 11-15, 27-31, 53-57, 68
%  \end{enumerate}
%
%  \vfill
%  Atenção: A prova é baseada no livro, não nas apresentações
%\end{frame}

%------------------------------------------------------------------------------%
\end{document}
%------------------------------------------------------------------------------%
