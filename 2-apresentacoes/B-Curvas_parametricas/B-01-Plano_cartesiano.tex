%------------------------------------------------------------------------------%
\documentclass[fleqn,utf8,aspectratio=169,12pt,ignorenonframetext]{beamer}

\usepackage{calculo_varias_variaveis-1}

\usepackage{tikz}
\usetikzlibrary{angles}

\title{Plano Cartesiano}

%------------------------------------------------------------------------------%
\begin{document}

\addtitle

%------------------------------------------------------------------------------%
\section{Sistema de Coordenadas Cartesianas}

%------------------------------------------------------------------------------%
\begin{frame}{Sistema de Coordenadas Cartesianas}
  \begin{enumerate}[<+->]
    \item Cada ponto do plano é associado a um par de números reais $(x, y)$
    \item Cada ``figura'' no plano corresponde a um subconjunto de $\R^2$
    \item Na orientação convencional, $x$ aponta para a direita e $y$ para cima
  \end{enumerate}
\end{frame}

\framedgraphic{Teste}{transformacao}{}

%------------------------------------------------------------------------------%
\end{document}
%------------------------------------------------------------------------------%
