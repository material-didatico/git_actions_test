%------------------------------------------------------------------------------%
\documentclass[a4paper,12pt,fleqn]{article}

\usepackage{calculo_varias_variaveis-1}

\ShowAnswers

%------------------------------------------------------------------------------%
\begin{document}

\makeHeader{CFVVI}{Prova 3 -- Turma 2}{14/07/2025}

\setlength{\jot}{6pt}

% 01
%------------------------------------------------------------------------------%
\exercise{20}
Calcule raízes cúbicas do número complexo \( z = 1 + i\sqrt{3} \)
\clearpagequestiononly

\begin{answer}
  Converter para forma polar
  \begin{align*}
    x       & = \Re(z) = 1        \qquad\qquad
    y         = \Im(z) = \sqrt{3} \qquad\qquad
    |z|       = \sqrt{x^2+y^2}
              = \sqrt{1^2 + \left(\sqrt{3}\right)^2}
              = 2         \\
    \varphi & = \arg(z)
              = \arctan\left(\frac{y}{x}\right)
              = \arctan\left(\frac{\sqrt{3}}{1}\right)
              = \frac{\pi}{3} \\
    z       & = 2 \cis{\frac{\pi}{3}}
              = 2e^{i\pi/3}
  \end{align*}

  Aplicar a fórmula de De Moivre para as raízes cúbicas
  \[
    u_k = \sqrt[3]{2}\cis{\frac{\pi/3 + 2k\pi}{3}}
    \qquad\qquad k = 0,1,2
  \]
  Calculando os argumentos das raízes
  \begin{align*}
    \varphi_0
    & = \frac{\pi/3 + 2\times 0\pi}{3}
      = \frac{\pi}{9} \\
    \varphi_1
    & = \frac{\pi/3 + 2\times 1\pi}{3}
      = \frac{\pi/3 + 6\pi/3}{3}
      = \frac{7\pi}{9} \\
    \varphi_2
    & = \frac{\pi/3 + 2\times 2\pi}{3}
      = \frac{\pi/3 + 12\pi/3}{3}
      = \frac{13\pi}{9}
  \end{align*}
  As raízes são
  \begin{align*}
    u_0 & = \sqrt[3]{2} \cis{\frac{\pi}{9}}   = \sqrt[3]{2} e^{\sfrac{  i\pi}{9}}\\
    u_1 & = \sqrt[3]{2} \cis{\frac{7\pi}{9}}  = \sqrt[3]{2} e^{\sfrac{7 i\pi}{9}}\\
    u_2 & = \sqrt[3]{2} \cis{\frac{13\pi}{9}} = \sqrt[3]{2} e^{\sfrac{13i\pi}{9}}
  \end{align*}
\end{answer}

% 02
%------------------------------------------------------------------------------%
\exercise{20}
Converta a equação para coordenadas cartesianas e mostre que a solução é uma cônica
\[
  r = \frac{10}{4 + \sin\theta}
\]
\clearpagequestiononly

\begin{answer}
  Queremos mostrar que a equação, escrita em coordenadas cartesianas, $(x, y)$,
  é uma expressão com termos de segundo grau em $x$ e $y$
  \begin{align*}
    r                         & = \frac{10}{4 + \sin\theta} \\
    r(4 + \sin\theta)         & = 10                        \\
    4r + r\sin\theta          & = 10                        \\
    4r + y                    & = 10                        \\
    4r                        & = 10 - y                    \\
    16\left(x^2 + y^2\right)  & = 100 - 20y + y^2           \\
    16x^2 + 16y^2             & = 100 - 20y + y^2           \\
    16x^2 + 15y^2 + 20y - 100 & = 0
  \end{align*}
\end{answer}

% 03
%------------------------------------------------------------------------------%
\exercise{30}
Encontre e classifique os pontos críticos da função
\(
  f(x, y) = \sin(x)\sin(y)
\)
na região aberta $-\pi<x<\pi$ e  $-\pi<y<\pi$
\clearpagequestiononly

\begin{answer}
  \setlength\columnsep{3em}
  \begin{multicols}{2}
    \raggedcolumns
    Calculando as \textbf{derivadas primeiras} de $f$
    \begin{align*}
      \D{f}{x}
      & = \D{}{x}\left(\sin(x)\sin(y)\right) \\
      & = \D{}{x}\left(\sin(x)\right)\sin(y) \\
      & = \cos(x)\sin(y)
    \end{align*}
    \begin{align*}
      \D{f}{y}
      & = \D{}{y}\left(\sin(x)\sin(y)\right) \\
      & = \sin(x)\D{}{y}\left(\sin(y)\right) \\
      & = \sin(x)\cos(y)
    \end{align*}
    Encontrando os \textbf{pontos críticos}.
    As derivadas parciais de $f$ existem em todo o plano,
    temos então que resolver o sistema \(\nabla f = 0\)
    \begin{align*}
      \cos(x)\sin(y) & = 0 \\
      \sin(x)\cos(y) & = 0
    \end{align*}
    Da primeira equação temos que
    \[
      \cos(x) = 0 \qquad x = \pm\frac{\pi}{2}
    \]
    ou
    \[
      \sin(y) = 0 \qquad y = 0
    \]
    Se $y=0$ a segunda equação se torna
    \begin{align*}
      \sin(x)\cos(0) & = 0 \\
      \sin(x)        & = 0 \\
      x              & = 0
    \end{align*}
    Então o primeiro ponto é
    \[
      P_1 = (0, 0)
    \]

    \vspace{\baselineskip}
    Se $x = \sfrac{\pi}{2}$ a segunda equação se torna
    \begin{align*}
      \sin\left(\frac{\pi}{2}\right)\cos(y) & = 0                \\
      \cos(y)                               & = 0                \\
      y                                     & = \pm\frac{\pi}{2}
    \end{align*}
    Temos então os pontos
    \[
      P_2 = \left(\frac{\pi}{2}, \frac{\pi}{2}\right)
      \quad\text{e}\quad
      P_3 = \left(\frac{\pi}{2}, -\frac{\pi}{2}\right)
    \]

    \vspace{\baselineskip}
    Se $x = -\sfrac{\pi}{2}$ a segunda equação se torna
    \begin{align*}
      \sin\left(-\frac{\pi}{2}\right)\cos(y) & = 0                \\
      -\cos(y)                               & = 0                \\
      y                                      & = \pm\frac{\pi}{2}
    \end{align*}
    Temos então os pontos
    \[
      P_4 = \left(-\frac{\pi}{2}, \frac{\pi}{2}\right)
      \quad\text{e}\quad
      P_5 = \left(-\frac{\pi}{2}, -\frac{\pi}{2}\right)
    \]

    \vspace{\baselineskip}
    Para \textbf{classificar} os pontos críticos precisamos
    do determinante da Hessiana
    \begin{align*}
      \DS{f}{x}
      & = \D{}{x}\left(\cos(x)\sin(y)\right)
        = -\sin(x)\sin(y) \\
      \DS{f}{y}
      & = \D{}{y}\left(\sin(x)\cos(y)\right)
        = -\sin(x)\sin(y) \\
      \DM{f}{x}{y}
      & = \D{}{x}\left(\sin(x)\cos(y)\right)
        = \cos(x)\cos(y)
    \end{align*}
    \begin{align*}
      D(x, y)
      & = f_{xx}(x, y) f_{yy}(x, y) - \left(f_{xy}(x, y)\right)^2 \\
      & = \left(-\sin(x)\sin(y)\right) \\
      & \hpm \times
          \left(-\sin(x)\sin(y)\right) \\
      & \hpm - \left(\cos(x)\cos(y)\right)^2 \\
      & = \sin^2(x)\sin^2(y) - \cos^2(x)\cos^2(y)
    \end{align*}

    \vspace{\baselineskip}
    Classificando o ponto $P_1$
    \begin{align*}
      D\left(0, 0\right)
      & = \sin^2(0)\sin^2(0) - \cos^2(0)\cos^2(0) \\
      & = 0 \times 0 - 1 \times 1 = -1 < 0
    \end{align*}
    O ponto $P_1$ é um ponto de sela

    \vspace{\baselineskip}
    Classificando o ponto $P_2$
    \begin{align*}
      D\left(\frac{\pi}{2}, \frac{\pi}{2}\right)
      & = \sin^2\left(\frac{\pi}{2}\right)
          \sin^2\left(\frac{\pi}{2}\right) \\
      & - \cos^2\left(\frac{\pi}{2}\right)
          \cos^2\left(\frac{\pi}{2}\right) \\
      & = 1 \times 1 - 0 \times 0 \\
      &  = 1 > 0  \\
      f_{xx}\left(\frac{\pi}{2}, \frac{\pi}{2}\right)
      & = -\sin\left(\frac{\pi}{2}\right)
           \sin\left(\frac{\pi}{2}\right) \\
      & = -1\times 1 \\
      &  = -1 < 0
    \end{align*}
    O ponto $P_2$ é um máximo local

    \vspace{\baselineskip}
    Classificando o ponto $P_3$
    \begin{align*}
      D\left(\frac{\pi}{2}, -\frac{\pi}{2}\right)
      & = \sin^2\left( \frac{\pi}{2}\right)
          \sin^2\left(-\frac{\pi}{2}\right) \\
      & - \cos^2\left( \frac{\pi}{2}\right)
          \cos^2\left(-\frac{\pi}{2}\right) \\
      & = 1 \times (-1)^2 - 0 \times 0 \\
      &  = 1 > 0  \\
      f_{xx}\left(\frac{\pi}{2},-\frac{\pi}{2}\right)
      & = -\sin\left( \frac{\pi}{2}\right)
           \sin\left(-\frac{\pi}{2}\right) \\
      & = -1\times (-1) \\
      &  = 1 > 0
    \end{align*}
    O ponto $P_3$ é um mínimo local

    \vspace{\baselineskip}
    Classificando o ponto $P_4$
    \begin{align*}
      D\left(-\frac{\pi}{2}, \frac{\pi}{2}\right)
      & = \sin^2\left(-\frac{\pi}{2}\right)
          \sin^2\left( \frac{\pi}{2}\right) \\
      & - \cos^2\left(-\frac{\pi}{2}\right)
          \cos^2\left( \frac{\pi}{2}\right) \\
      & = (-1)^2 \times 1 - 0 \times 0 \\
      &  = 1 > 0  \\
      f_{xx}\left(-\frac{\pi}{2}, \frac{\pi}{2}\right)
      & = -\sin\left(-\frac{\pi}{2}\right)
           \sin\left( \frac{\pi}{2}\right) \\
      & = -(-1)\times 1 \\
      &  = 1 > 0
    \end{align*}
    O ponto $P_4$ é um mínimo local

    \vspace{\baselineskip}
    Classificando o ponto $P_5$
    \begin{align*}
      D\left(-\frac{\pi}{2}, -\frac{\pi}{2}\right)
      & = \sin^2\left(-\frac{\pi}{2}\right)
          \sin^2\left(-\frac{\pi}{2}\right) \\
      & - \cos^2\left(-\frac{\pi}{2}\right)
          \cos^2\left(-\frac{\pi}{2}\right) \\
      & = (-1)^2 \times (-1)^2 - 0 \times 0 \\
      & = 1 > 0  \\
      f_{xx}\left(-\frac{\pi}{2},-\frac{\pi}{2}\right)
      & = -\sin\left(-\frac{\pi}{2}\right)
           \sin\left(-\frac{\pi}{2}\right) \\
      & = -(-1) \times (-1) \\
      & = -1 < 0
    \end{align*}
    O ponto $P_5$ é um máximo local
  \end{multicols}
\end{answer}

% 04
%------------------------------------------------------------------------------%
\exercise{30}
Encontre o valor mínimo de
\(
  x^2 + y^2
\)
sujeito a
\(
  xy = 1
\)
\clearpagequestiononly

\begin{answer}
\setlength\columnsep{3em}
\begin{multicols}{2}
\raggedcolumns
  Queremos encontrar os valores máximo e mínimo da função
  \[
    f(x,y) = x^2 + y^2
  \]
  sujeitos a restrição
  \[
    g(x, y) = xy = 1
  \]
  Gradiente de $f$
  \[
    \D{f}{x}
    = \D{}{x}\left(x^2 + y^2\right)
    = 2x
  \]
  \[
    \D{f}{y}
    = \D{}{y}\left(x^2 + y^2\right)
    = 2y
  \]
  \[
    \nabla f(x, y) = \vetor{2x}{2y}
  \]
  Gradiente de $g$
  \[
    \D{g}{x}
    = \D{}{x}\left(xy\right)
    = y
  \]
  \[
    \D{g}{y}
    = \D{}{y}\left(xy\right)
    = x
  \]
  \[
    \nabla g(x, y) = \vetor{y}{x}
  \]
  Precisamos resolver o sistema
  \begin{align*}
    2x & = \lambda y \\
    2y & = \lambda x \\
    xy & = 1
  \end{align*}
  Pela terceira equação sabemos que $x\neq 0$ e $y\neq 0$.
  Isolando $\lambda$ na primeira, \(\lambda = \frac{2x}{y}\),
  e substituindo na segunda, temos
  \begin{align*}
    2y  & = \lambda x      \\
    2y  & = \frac{2x}{y} x \\
    y^2 & = x^2            \\
    y   & = \pm x
  \end{align*}

  Se $y=x$, da terceira equação temos
  \begin{align*}
    xy  & = 1     \\
    x^2 & = 1     \\
    x   & = \pm 1
  \end{align*}
  Temos, então, os pontos
  \[
    P_1 = (1, 1)
    \qquad\text{e}\qquad
    P_2 = (-1, -1)
  \]

  Se $y=-x$, da terceira equação temos
  \begin{align*}
    xy    & = 1   \\
    x(-x) & = 1   \\
    -x^2  & = 1   \\
    x^2   & = - 1
  \end{align*}
  Não existe solução neste caso.

  Avaliando $f$ nos pontos encontrados, temos
  \begin{align*}
    f(1, 1)   & = 1^2 + 1^2 = 2      \\
    f(-1, -1) & = (-1)^2 + (-1)^2= 2
  \end{align*}
  Os dois pontos são pontos de mínimo e o valor mínimo da função é 2
\end{multicols}
\end{answer}

%------------------------------------------------------------------------------%
\end{document}
%------------------------------------------------------------------------------%
