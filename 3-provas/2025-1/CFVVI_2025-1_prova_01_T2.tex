%------------------------------------------------------------------------------%
\documentclass[a4paper,12pt,fleqn]{article}

\usepackage{calculo_varias_variaveis-1}

\ShowAnswers

%------------------------------------------------------------------------------%
\begin{document}

\makeHeader{CFVVI}{Prova 1 -- Turma 2}{05/05/2025}

% 01
%------------------------------------------------------------------------------%
\exercise{25}
Encontre uma parametrização para o círculo definido pela equação
\( (x-3)^2 + (y+1)^2 = 4\)
\clearpagequestiononly

\begin{answer}
  Círculo de raio 2 centrado no ponto \((3, -1)\)

  Parametrização do círculo unitário centrado na origem
  \begin{align*}
    x      & = \cos(\theta) \\
    y      & = \sin(\theta) \\
    \theta & \in [0, 2\pi)
  \end{align*}
  Parametrização do círculo de raio 2 centrado na origem
  \begin{align*}
    x      & = 2\cos(\theta) \\
    y      & = 2\sin(\theta) \\
    \theta & \in [0, 2\pi)
  \end{align*}
  Parametrização do círculo de raio 2 centrado no ponto \((3, -1)\)
  \begin{align*}
  x      & = 2\cos(\theta) + 3 \\
  y      & = 2\sin(\theta) - 1 \\
  \theta & \in [0, 2\pi)
  \end{align*}
\end{answer}

% 02
%------------------------------------------------------------------------------%
\exercise{25}
Considerando a função
\(
  f(x,y) = \sqrt{16 - (x-1)^2 - (y-1)^2}
\)
\begin{enumerate}[label=(\alph*)]
  \item \ [10] Determine o domínio de $f$ e represente-o graficamente
  \item \ [5] Encontre a imagem de $f$
  \item \ [10] Caracterize todas as curvas de nível de $f$ e esboce três delas
\end{enumerate}
\begin{questiononly}
\begin{multicols}{2}
  Domínio
  \vspace{-0.8\baselineskip}
  \begin{center}
    \includegraphics{axis_xy}
  \end{center}
  Curvas de nível
  \vspace{-0.8\baselineskip}
  \begin{center}
    \includegraphics{axis_xy}
  \end{center}
\end{multicols}
\end{questiononly}
\clearpagequestiononly

\begin{answer}
\begin{multicols}{2}
  Domínio
  \vspace{-0.8\baselineskip}
  \begin{center}
    \includegraphics{P1_T2_dominio}
  \end{center}
  Curvas de nível
  \vspace{-0.8\baselineskip}
  \begin{center}
    \includegraphics{P1_T2_curvas_nivel}
  \end{center}
\end{multicols}

\begin{enumerate}[label=(\alph*)]
\item O domínio de $f$ é o conjunto de todos os pontos \((x, y)\) tais que:
  \[
    16 - (x-1)^2 - (y-1)^2 \geq 0
  \]
  \[
    (x-1)^2 + (y-1)^2 \leq 4^2
  \]
  ou seja, os pontos do circulo centrado em \((1, 1)\) de raio 4.
  Portanto, o domínio é
  \[
    D_f = \{(x, y) \in \R^2 \st (x-1)^2 + (y-1)^2 \leq 4^2\}
  \]

\item A imagem de $f$ é o conjunto de valores que $f$ pode assumir.
  Como \( (x-1)^2 + (y-1)^2 \leq 16 \), temos:
  \[
    0 \leq \sqrt{16 - (x-1)^2 - (y-1)^2} \leq 4
  \]
  Logo, a imagem é:
  \[
    \text{Im}(f) = [0, 4]
  \]

\item As curvas de nível são definidas por $f(x,y) = c$ para $c \in [0, 4]$.
  \begin{gather*}
    \sqrt{16 - (x-1)^2 - (y-1)^2} = c \\
    (x-1)^2 + (y-1)^2 = 16 - c^2
  \end{gather*}
  Cada curva de nível é uma circunferência com
  centro em $(1, 1)$ e raio $\sqrt{16 - c^2}$

  \begin{align*}
    r & = 4 & c & = \sqrt{16 - r^2} = 0         & (x-1)^2 + (y-1)^2 & = 16 \\
    r & = 2 & c & = \sqrt{16 - r^2} = \sqrt{12} & (x-1)^2 + (y-1)^2 & = 4  \\
    r & = 1 & c & = \sqrt{16 - r^2} = \sqrt{15} & (x-1)^2 + (y-1)^2 & = 1
  \end{align*}
\end{enumerate}
\end{answer}

% 03
%------------------------------------------------------------------------------%
\exercise{30}
Calcule o limite ou mostre que ele não existe
\begin{enumerate}[label=(\alph*)]
  \item \ [15] \(\lim_{(x,y) \to (0,0)} \frac{xy}{x^2 + y^2}\)
  \item \ [15] \(\lim_{(x,y) \to (0,0)} \frac{\sqrt{x^2 + y^2 + 1} - 1}{x^2 + y^2} \)
\end{enumerate}
\clearpagequestiononly

\begin{answer}
\begin{enumerate}[label=(\alph*)]
  \item Testando diferentes caminhos \\
  Caminho \(y = x\)
  \[
    \lim_{(x,y) \to (0,0)} \frac{xy}{x^2 + y^2} \evalat{y=x}{}
    = \lim_{x \to 0} \frac{x x}{x^2 + x^2}
    = \lim_{x \to 0} \frac{x^2}{2x^2}
    = \lim_{x \to 0} \frac{1}{2}
    = \frac{1}{2}
  \]
  Caminho \(y = -x\)
  \[
    \lim_{(x,y) \to (0,0)} \frac{xy}{x^2 + y^2} \evalat{y=-x}{}
    = \lim_{x \to 0} \frac{x(-x)}{x^2 + x^2}
    = \lim_{x \to 0} \frac{-x^2}{2x^2}
    = \lim_{x \to 0} \frac{-1}{2}
    = -\frac{1}{2}
  \]
  Como os limites são diferentes, o limite \textbf{não existe}

\item Multiplicando pelo conjugado
  \begin{align*}
    \frac{\sqrt{x^2 + y^2 + 1} - 1}{x^2 + y^2}
    & = \frac{\sqrt{x^2 + y^2 + 1} - 1}{x^2 + y^2}
        \times
        \frac{\sqrt{x^2 + y^2 + 1} + 1}{\sqrt{x^2 + y^2 + 1} + 1} \\
    & = \frac{x^2 + y^2}{(x^2 + y^2)(\sqrt{x^2 + y^2 + 1} + 1)} \\
    & = \frac{1}{\sqrt{x^2 + y^2 + 1} + 1}
  \end{align*}
  \begin{align*}
    \lim_{(x,y) \to (0,0)} \frac{\sqrt{x^2 + y^2 + 1} - 1}{x^2 + y^2}
    & = \lim_{(x,y) \to (0,0)} \frac{1}{\sqrt{x^2 + y^2 + 1} + 1}  \\
    & = \frac{1}{\sqrt{0^2 + 0^2 + 1} + 1}   \\
    & = \frac{1}{2}
  \end{align*}
\end{enumerate}
\end{answer}

% 04
%------------------------------------------------------------------------------%
\exercise{20}
Encontre e esboce as curvas que representam os cortes da superfície
\(y^2-x^2=z\)
pelos planos
\(z=0\) e
\(y=0\)
\begin{questiononly}
  \begin{multicols}{2}
  \(z=0\)
  \vspace{-0.8\baselineskip}
  \begin{center}
    \includegraphics{axis_xy}
  \end{center}
  \(y=0\)
  \vspace{-0.8\baselineskip}
  \begin{center}
    \includegraphics{axis_xz}
  \end{center}
\end{multicols}
\end{questiononly}

\begin{answer}
  \begin{multicols}{2}
    \(z=0\)
    \vspace{-0.8\baselineskip}
    \begin{center}
      \includegraphics{T2_conica_z0}
    \end{center}
    \(y=0\)
    \vspace{-0.8\baselineskip}
    \begin{center}
      \includegraphics{T2_conica_y0}
    \end{center}
  \end{multicols}

  No plano \(z=0\) a equação \(y^2-x^2=z\) se reduz a
  \begin{align*}
    y^2-x^2 & = 0       \\
    y^2     & = x^2     \\
    \abs{y} & = \abs{x} \\
    y       & = \pm x
  \end{align*}
  que corresponde as retas \(y=x\) e \(y=-x\)

  No plano \(y=0\) a equação \(y^2-x^2=z\) se reduz a
  \begin{align*}
    0-x^2 & = z    \\
    z     & = -x^2
  \end{align*}
  que corresponde uma parábola com vértice na origem e apontando para baixo
  no plano $zx$
\end{answer}

%------------------------------------------------------------------------------%
\end{document}
%------------------------------------------------------------------------------%
