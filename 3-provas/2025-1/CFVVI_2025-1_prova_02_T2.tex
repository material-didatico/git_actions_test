%------------------------------------------------------------------------------%
\documentclass[a4paper,12pt,fleqn]{article}

\usepackage{calculo_varias_variaveis-1}

\ShowAnswers

%------------------------------------------------------------------------------%
\begin{document}

\makeHeader{CFVVI}{Prova 2 -- Turma 2}{04/06/2024}

% 01
%------------------------------------------------------------------------------%
\exercise{25}
Utilize a linearização da função
\(
  f(x, y) = e^{x^2 - y^2} \cos(xy)
\)
no ponto \(\left(\sqrt{\pi}, \sqrt{\pi}\right)\),
para obter uma estimativa para \(f(2, 1)\)
\clearpagequestiononly

\begin{answer}
  Calculando as derivadas parciais
  \begin{align*}
    \D{f}{x} & = \hpm e^{x^2 - y^2} \big(2x \cos(xy) - y \sin(xy)\big) \\
    \D{f}{y} & =    - e^{x^2 - y^2} \big(2y \cos(xy) + x \sin(xy)\big)
  \end{align*}
  Avaliando as derivadas parciais no ponto \(\left(\sqrt{\pi}, \sqrt{\pi}\right)\)
  \begin{align*}
    \D{f}{x}\left(\sqrt{\pi}, \sqrt{\pi}\right)
      & = e^{\sqrt{\pi}^2 - \sqrt{\pi}^2}
          \Big(2\sqrt{\pi} \cos\left(\sqrt{\pi}\sqrt{\pi}\right)
              - \sqrt{\pi} \sin\left(\sqrt{\pi}\sqrt{\pi}\right)\Big) \\
      & = e^{0}
          \Big(2\sqrt{\pi} \cos\left(\pi\right)
              - \sqrt{\pi} \sin\left(\pi\right)\Big) \\
      & = 2\sqrt{\pi} (-1) - \sqrt{\pi} 0 \\
      & = -2\sqrt{\pi} \\
    \D{f}{y}\left(\sqrt{\pi}, \sqrt{\pi}\right)
      & = - e^{\sqrt{\pi}^2 - \sqrt{\pi}^2}
          \Big(2\sqrt{\pi} \cos(\sqrt{\pi}\sqrt{\pi})
              + \sqrt{\pi} \sin(\sqrt{\pi}\sqrt{\pi})\Big) \\
      & = - e^{0} \Big(2\sqrt{\pi} (-1) + \sqrt{\pi} 0\Big) \\
      & = 2\sqrt{\pi}
  \end{align*}
  Avaliando a função no ponto  \(\left(\sqrt{\pi}, \sqrt{\pi}\right)\)
  \begin{align*}
    f(\sqrt{\pi}, \sqrt{\pi})
      & = e^{x^2 - y^2} \cos(xy) \\
      & = e^{\sqrt{\pi}^2 - \sqrt{\pi}^2} \cos\left(\sqrt{\pi}\sqrt{\pi}\right) \\
      & = e^{0} \cos\left(\pi\right) \\
      & = -1
  \end{align*}
  Escrevendo a aproximação linear centrada no ponto  \(\left(\sqrt{\pi}, \sqrt{\pi}\right)\)
  \begin{align*}
    L(x, y)
      & = f(x_0, y_0) + f_x(x_0, y_0)(x-x_0) + f_y(x_0, y_0)(y-y_0) \\
      & = f\left(\sqrt{\pi}, \sqrt{\pi}\right)
        + f_x\left(\sqrt{\pi}, \sqrt{\pi}\right)\left(x-\sqrt{\pi}\right)
        + f_y\left(\sqrt{\pi}, \sqrt{\pi}\right)\left(y-\sqrt{\pi}\right) \\
      & = -1
        - 2\sqrt{\pi}\left(x-\sqrt{\pi}\right)
        + 2\sqrt{\pi}\left(y-\sqrt{\pi}\right)  \\
      & = -1 - 2\sqrt{\pi}x + 2\pi + 2\sqrt{\pi}y - 2\pi \\
      & = 2\sqrt{\pi}(y-x) -1
  \end{align*}
  Avaliando a estimativa para \(f(3, 2)\)
  \[
    f(2, 1) \approx L(2, 1)
            = 2\sqrt{\pi}(1-2) -1
            = -2\sqrt{\pi} -1
            \approx -4.54490770181
  \]
  Comparando com o valor exato
  \[
    f(2, 1)
    = e^3\cos(2)
    \approx -8.35853265094
  \]
  os pontos estão muito longe para que a aproximação linear seja adequada
\end{answer}

% 02
%------------------------------------------------------------------------------%
\exercise{25}
Considere que a equação
\(
  \sin(xz) + y^2 z = 1
\)
defina $z$ como função de $x$ e $y$, encontre \(\D{z}{x}\)
\clearpagequestiononly

\begin{answer}
  \begin{align*}
    \sin(xz) + y^2 z                                   & = 1                                    \\
    \D{}{x}\left(\sin(xz) + y^2 z\right)               & = 0                                    \\
    \cos(xz)\left(z + x \D{z}{x}\right) + y^2 \D{z}{x} & = 0                                    \\
    z\cos(xz) + \left(x\cos(xz) + y^2\right) \D{z}{x}  & = 0                                    \\
    \left( x \cos(xz) + y^2 \right) \D{z}{x}           & = -z \cos(xz)                          \\
    \D{z}{x}                                           & = \frac{-z \cos(xz)}{x \cos(xz) + y^2}
  \end{align*}
\end{answer}

% 03
%------------------------------------------------------------------------------%
\exercise{25}
Seja \( f(x, y) = x^2 y + \cos(y) \)
\begin{enumerate}[label=\alph*)]
  \item Calcule o gradiente de $f$
  \item Calcule a derivada de $f$ no ponto $(1, \pi)$ na direção do vetor $\vetor{3}{4}$
\end{enumerate}
\clearpagequestiononly

\begin{answer}
  \begin{enumerate}[label=\alph*)]
    \item
      Vetor gradiente
      \[
        \nabla f(x, y)
        = \Vetor{\D{f}{x}}{\D{f}{y}}
        = \vetor{2x y}{x^2 - \sin(y)}
      \]
    \item
      Gradiente no ponto $(1, \pi)$
      \[
        \nabla f(x, y) (1, \pi)
          = \vetor{2\times 1 \times\pi}{1^2 - \sin(\pi)}
          = \vetor{2\pi}{1}
      \]
      Calculando um vetor unitário na direção de $v$
      \[
        u
        = \frac{v}{\norm{v}}
        = \frac{1}{\sqrt{3^2 + 4^2}} \vetor{3}{4}
        = \frac{1}{5} \vetor{3}{4}
      \]
      Derivada direcional
      \begin{align*}
        D_u f(1, \pi) & = \nabla f(1, \pi) \cdot u                       \\
                      & = \vetor{2\pi}{1} \cdot \frac{1}{5} \vetor{3}{4} \\
                      & = \frac{1}{5} (6\pi + 4) \\
                      & = \frac{6\pi + 4}{5}
      \end{align*}
  \end{enumerate}
\end{answer}

% 04
%------------------------------------------------------------------------------%
\exercise{25}
Seja \( f(x, y, z) = \ln(1 + x y^2 + z^2) \),
calcule \(\DM{f}{x}{z} \) no ponto \( (1, 1, 2) \).
Efetue as derivadas na ordem especificada pela notação.
\clearpagequestiononly

\begin{answer}
\begin{align*}
  \D{f}{z}
  & = \D{}{z} \ln(1 + x y^2 + z^2) \\
  & = \frac{1}{1 + x y^2 + z^2} \D{}{z}(1 + x y^2 + z^2) \\
  & = \frac{2z}{1 + x y^2 + z^2}
\end{align*}
\begin{align*}
  \DM{f}{x}{z}
  & = \D{}{x}\left(\frac{2z}{1 + x y^2 + z^2}\right) \\
  & = 2z\D{}{x}\left(1 + x y^2 + z^2\right)^{-1} \\
  & = 2z(-1)\left(1 + x y^2 + z^2\right)^{-2}\D{}{x}\left(1 + x y^2 + z^2\right) \\
  & = \frac{-2z}{\left(1 + x y^2 + z^2\right)^2} \; y^2 \\
  & = \frac{-2z y^2}{(1 + x y^2 + z^2)^2}
\end{align*}
\begin{align*}
  \DM{f}{x}{z}(1, 1, 2)
  & = \frac{-2z y^2}{(1 + x y^2 + z^2)^2} \evalat{(1, 1, 2)}{} \\
  & = \frac{-2\times 2\times 1^2}{(1 + 1\times 1^2 + 2^2)^2}  \\
  & = \frac{-2^2}{6^2}  \\
  & = \frac{-2^2}{2^2\times 3^2}  \\
  & = \frac{-1}{9}
\end{align*}
\end{answer}

%------------------------------------------------------------------------------%
\end{document}
%------------------------------------------------------------------------------%
