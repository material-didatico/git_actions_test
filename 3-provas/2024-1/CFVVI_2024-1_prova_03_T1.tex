%------------------------------------------------------------------------------%
% Luis A. D'Afonseca
%
%------------------------------------------------------------------------------%
\documentclass[a4paper,12pt,fleqn]{article}

\usepackage{style_avaliacao}

\ShowAnswers

\newcommand{\D}[2]{\frac{\partial {#1}}{\partial {#2}}}
\newcommand{\DS}[2]{\frac{\partial^2 {#1}}{\partial {#2}^2}}
\newcommand{\DM}[3]{\frac{\partial^2 {#1}}{\partial {#2} \partial {#3}}}

\newcommand{\vetor}[2]{\left(\begin{array}{c} #1 \\ #2 \end{array}\right)}

\newcommand{\veci}{\bm{i}}
\newcommand{\vecj}{\bm{j}}
\newcommand{\veck}{\bm{k}}

\newcommand{\comentario}[1]{{\color{magenta} #1 \par\vspace{\baselineskip}\par}}

%------------------------------------------------------------------------------%
\begin{document}

\makeHeader{CFVVI}{Prova 3 -- Turma 1}{09/09/2024}

%\comentario{
%  Nos exercícios sobre números complexos:
%  \begin{itemize}
%    \item Como esta é uma disciplina de cálculo, exigi que as respostas sejam sempre
%    em radianos, mas os alunos podem fazer os passos intermediários em graus.
%
%    \item No cálculo das raizes o aluno pode deixar a resposta final na forma polar,
%    não é necessário retornar para a forma canônica.
%  \end{itemize}
%}

% 01
%------------------------------------------------------------------------------%
\exercise{50}
Encontre o ponto da superfície \(z=xy+1\) que está mais próximo da origem.

\begin{answer}
  Queremos minimizar a distância
  \[
    d(x, y, z) = \sqrt{x^2 + y^2 + z^2}
  \]
  Como a função raiz quadrada é crescente podemos minimizar a função
  \[
    f(x, y, z) = x^2 + y^2 + z^2
  \]
  suas derivadas parciais são
  \[
    f_x(x, y, z) = 2x
    \qquad
    f_y(x, y, z) = 2y
    \qquad
    f_z(x, y, z) = 2z
  \]
  As derivadas parciais da função \(g(x, y, z)=z-xy\) são
  \[
    g_x(x, y, z) = -y
    \qquad
    g_y(x, y, z) = -x
    \qquad
    g_z(x, y, z) = 1
  \]
  Aplicando os Multiplicadores de Lagrange, precisamos resolver o sistema
  \[
  \begin{cases}
    2x = -\lambda y \\
    2y = -\lambda x \\
    2z = \lambda  \\
    z-xy = 1
  \end{cases}
  \]
  Isolamos $x$ na primeira equação e substituimos na segunda
  \[
    x = \frac{-\lambda y}{2}
  \]
  \[
    y
    = \frac{-\lambda}{2} x
    = \frac{-\lambda}{2} \; \frac{-\lambda y}{2}
    = \frac{\lambda^2}{4} y
  \]
  Temos $y=0$ ou
  \[
    \frac{\lambda^2}{4} = 1
    \quad\Rightarrow\quad
    \lambda^2 = 4
    \quad\Rightarrow\quad
    \lambda = \pm 2
  \]

  Se $y=0$ a quarta equação se reduz a $z=1$, pela terceira \(\lambda=2\) e
  \[
    x
    = \frac{-\lambda y}{2}
    = \frac{-2\times 0}{2}
    = 0
  \]
  Obtemos o ponto \((x_1, y_1, z_1) = (0, 0, 1)\)

  \vspace{2\baselineskip}
  Se $\lambda=2$
  \[
    x
    = \frac{-\lambda y}{2}
    = \frac{-2y}{2}
    = -y
  \]
  \[
    2z = \lambda = 2
    \quad\Rightarrow\quad
    z = 1
  \]
  Substituindo na quarta equação
  \begin{align*}
    z-xy & = 1 \\
    1 - (-y)y & = 1 \\
    y^2 = 0 \\
    y = 0
  \end{align*}
  Obtemos novamente o ponto \((x_1, y_1, z_1) = (0, 0, 1)\)

  \vspace{2\baselineskip}
  Se $\lambda=-2$
  \[
    x
    = \frac{-\lambda y}{2}
    = \frac{-(-2)y}{2}
    = y
  \]
  \[
    2z = \lambda = -2
    \quad\Rightarrow\quad
    z = -1
  \]
  Substituindo na quarta equação
  \begin{align*}
    z-xy & = 1 \\
    -1 - y^2 & = 1 \\
    -y^2 = 2 \\
    y^2 = -2
  \end{align*}
  Não existe solução real.

  O ponto mais próximo da origem é \((0, 0, 1)\).
  \clearpage
\end{answer}

% 02
%------------------------------------------------------------------------------%
\exercise{25}
Dado \(z=\left(\sqrt{3}+i\right)^4 \), calcule
\begin{enumerate}[label=\alph*),itemsep=0pt]
  \item a parte real de $z$,
  \item a parte imaginária de $z$,
  \item o módulo de $z$,
  \item o argumento de $z$
\end{enumerate}

\begin{answer}
  Convertendo $u=\sqrt{3}+i$ para a forma polar
  \[
    \rho
    = \abs{u}
    = \sqrt{\left(\sqrt{3}\right)^2 + 1^2}
    = \sqrt{4}
    = 2
  \]
  \[
    \sin(\varphi) = \frac{\Im(u)}{\abs{u}} = \frac{1}{2}
    \qquad\qquad
    \cos(\varphi) = \frac{\Re(u)}{\abs{u}} = \frac{\sqrt{3}}{2}
    \qquad\qquad
    \varphi = \arg(u) = \ang{30} = \frac{\pi}{6}
  \]
  Assim
  \[
    u
    = \rho\left[\cos\left(\varphi\right)+ i\sin\left(\varphi\right)\right]
    = 2\left[\cos\left(\frac{\pi}{6}\right)+ i\sin\left(\frac{\pi}{6}\right)\right]
  \]
  Portanto
  \begin{align*}
    z
    & = u^3 \\[2mm]
    & = \rho^4\left[\cos\left(4\varphi\right)+ i\sin\left(4\varphi\right)\right] \\[2mm]
    & = 2^4\left[\cos\left(\frac{4\pi}{6}\right)+ i\sin\left(\frac{4\pi}{6}\right)\right] \\[2mm]
    & = 16\left[\cos\left(\frac{2\pi}{3}\right)+ i\sin\left(\frac{2\pi}{3}\right)\right] \\[2mm]
    & = 16\left[-\frac{1}{2}+ i\frac{\sqrt{3}}{2}\right] \\[2mm]
    & = -8 + 8\sqrt{3}i
  \end{align*}
  Parte real de $z$
  \[
    \Re(z) = -8
  \]
  Parte imaginária de $z$
  \[
    \Im(z) = 8\sqrt{3}
  \]
  Módulo de $z$
  \[
    \abs{z} = 16
  \]
  Argumento de $z$
  \[
    \arg(z) = \frac{2\pi}{3}
  \]
  \clearpage
\end{answer}

% 03
%------------------------------------------------------------------------------%
\exercise{25}
Encontre as raízes cúbicas complexas do número \( z = 27i \).

\begin{answer}
  Escrevemos \( z = 27i \) na forma trigonométrica
  \[
    z = 27 \left[ \cos\left(\frac{\pi}{2}\right) + i \sin\left(\frac{\pi}{2}\right) \right]
  \]
  As raízes cúbicas são
  \begin{align*}
    u_k
    & = \sqrt[3]{27} \left[
              \cos\left(\frac{1}{3}\left(\frac{\pi}{2} + 2k\pi\right)\right)
          + i \sin\left(\frac{1}{3}\left(\frac{\pi}{2} + 2k\pi\right)\right)
        \right]
    & k = 0, 1, 2    \\[2mm]
    & = 3 \left[
            \cos\left( \frac{\pi + 4k\pi}{6} \right)
        + i \sin\left( \frac{\pi + 4k\pi}{6} \right)
        \right]
  \end{align*}
  Para cada valor de $k$
  \begin{align*}
    u_0
    & = 3 \left[
              \cos\left( \frac{\pi + 4\times0\pi}{6} \right)
          + i \sin\left( \frac{\pi + 4\times0\pi}{6} \right)
          \right] \\[2mm]
    & = 3 \left[
              \cos\left( \frac{\pi}{6} \right)
          + i \sin\left( \frac{\pi}{6} \right)
          \right] \\[2mm]
    & = 3 \left[
              \cos\left( \ang{30} \right)
          + i \sin\left( \ang{30} \right)
          \right] \\[2mm]
    & = 3 \left[
              \frac{\sqrt{3}}{2}
          + i \frac{1}{2}
          \right] \\[2mm]
    & = \frac{3\sqrt{3}}{2} + \frac{3}{2} i
  \end{align*}
  \begin{align*}
    u_1
    & = 3 \left[
              \cos\left( \frac{\pi + 4\times1\pi}{6} \right)
          + i \sin\left( \frac{\pi + 4\times1\pi}{6} \right)
          \right] \\[2mm]
    & = 3 \left[
              \cos\left( \frac{5\pi}{6} \right)
          + i \sin\left( \frac{5\pi}{6} \right)
          \right] \\[2mm]
    & = 3 \left[
              \cos\left( \ang{150} \right)
          + i \sin\left( \ang{150} \right)
          \right] \\[2mm]
    & = 3 \left[
            - \frac{\sqrt{3}}{2}
          + i \frac{1}{2}
          \right] \\[2mm]
    & = -\frac{3\sqrt{3}}{2} + \frac{3}{2}i
  \end{align*}
  \begin{align*}
    u_2
    & = 3 \left[
              \cos\left( \frac{\pi + 4\times2\pi}{6} \right)
          + i \sin\left( \frac{\pi + 4\times2\pi}{6} \right)
          \right] \\[2mm]
    & = 3 \left[
              \cos\left( \frac{9\pi}{6} \right)
          + i \sin\left( \frac{9\pi}{6} \right)
          \right] \\[2mm]
    & = 3 \left[
              \cos\left( \frac{3\pi}{2} \right)
          + i \sin\left( \frac{3\pi}{2} \right)
          \right] \\[2mm]
    & = 3 \left[
              \cos\left( \ang{270} \right)
          + i \sin\left( \ang{270} \right)
          \right] \\[2mm]
    & = 3 \left[
            0
            + i (-1)
          \right]
     = -3i
  \end{align*}
\end{answer}

%------------------------------------------------------------------------------%
\end{document}
%------------------------------------------------------------------------------%